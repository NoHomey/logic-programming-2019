\documentclass{article}[12pt]
\usepackage{amsmath,amsthm}
\usepackage{amssymb}
\usepackage{lipsum}
\usepackage{stmaryrd}
\usepackage[T1,T2A]{fontenc}
\usepackage[utf8]{inputenc}
\usepackage[bulgarian]{babel}
\usepackage[normalem]{ulem}
\usepackage{xcolor}

\newcommand{\Lang}{\mathcal{L}}

\setlength{\parindent}{0mm}

\title{Неопределимост}
\author{Иво Стратев}

\begin{document}

\maketitle

\tableofcontents

\pagebreak

\section{Хомоморфизми}

\subsection{Слаб хомоморфизъм}

Нека \(\Lang\) е език на предикатното смятане от първи ред.
Нека \(\mathcal{A}\) и \(\mathcal{B}\) са две структури за \(\Lang\).
Нека \(h \; : \; |\mathcal{A}| \to |\mathcal{B}|\). \\
\(h\) е слаб хомоморфизъм (\(h \in WeakHom(\mathcal{A}, \mathcal{B})\)),
ако

\begin{itemize}
\item \((\forall c \in Const_\Lang)[h(c^\mathcal{A}) = c^\mathcal{B}]\);
\item \((\forall f \in Func_\Lang)(\forall a_1 \in |\mathcal{A}|)\dots(\forall a_{\#f} \in |\mathcal{A}|)
[\; h(f^\mathcal{A}(a_1, \dots, a_{\#f})) = f^\mathcal{B}(h(a_1), \dots, h(a_{\#f}))\;]\);
\item \((\forall p \in Pred_\Lang)(\forall a_1 \in |\mathcal{A}|)\dots(\forall a_{\#p} \in |\mathcal{A}|) \newline
[<a_1, \dots, a_{\#p}> \; \in p^\mathcal{A} \longrightarrow <h(a_1), \dots, h(a_{\#p})> \; \in p^\mathcal{B}]\);
\end{itemize}

Накратко слабия хомоморфизъм запазва операциите (функциите) и запазва истинността на свойствата/връзките/релациите.

\subsection{Рефлектиращ хомоморфизъм или просто хомоморфизъм}

Нека \(\Lang\) е език на предикатното смятане от първи ред.
Нека \(\mathcal{A}\) и \(\mathcal{B}\) са две структури за \(\Lang\).
Нека \(h \; : \; |\mathcal{A}| \to |\mathcal{B}|\). \\
\(h\) е рефлектиращ (reflective) хомоморфизъм (\(h \in Hom(\mathcal{A}, \mathcal{B})\)),
ако

\begin{itemize}
\item \((\forall c \in Const_\Lang)[h(c^\mathcal{A}) = c^\mathcal{B}]\);
\item \((\forall f \in Func_\Lang)(\forall a_1 \in |\mathcal{A}|)\dots(\forall a_{\#f} \in |\mathcal{A}|)
[\; h(f^\mathcal{A}(a_1, \dots, a_{\#f})) = f^\mathcal{B}(h(a_1), \dots, h(a_{\#f}))\;]\);
\item \((\forall p \in Pred_\Lang)(\forall a_1 \in |\mathcal{A}|)\dots(\forall a_{\#p} \in |\mathcal{A}|) \newline
[<a_1, \dots, a_{\#p}> \; \in p^\mathcal{A} \longleftrightarrow \; <h(a_1), \dots, h(a_{\#p})> \; \in p^\mathcal{B}]\);
\end{itemize}

Забележа: Разликата е в последното свойство.

Накратко рефлектиращ хомоморфизъм или просто хомоморфизъм е слаб хомоморфизъм,
който запазва и лъжата (неистинността) на свойствата/връзките/релациите.

\subsection{Изоморфизъм}

Изоморфизъм е хомоморфизъм, който е и биекция!

\subsection{Автоморфизъм}

Автоморфизъм е изоморфизъм на една структура в себе си.
Но тъйкато понятието за нас е с особена важност ще го напишем!

\vspace{1cm}

Нека \(\Lang\) е език на предикатното смятане от първи ред.
Нека \(\mathcal{A}\) е структура за \(\Lang\).
Нека \(h \; : \; |\mathcal{A}| \to |\mathcal{A}|\). \\
\(h\) е автоморфизъм (\(h \in Aut(\mathcal{A})\)),
ако \(h\) е биекция и още:

\begin{itemize}
\item \((\forall c \in Const_\Lang)[h(c^\mathcal{A}) = c^\mathcal{A}]\);
\item \((\forall f \in Func_\Lang)(\forall a_1 \in |\mathcal{A}|)\dots(\forall a_{\#f} \in |\mathcal{A}|)
[\; h(f^\mathcal{A}(a_1, \dots, a_{\#f})) = f^\mathcal{A}(h(a_1), \dots, h(a_{\#f}))\;]\);
\item \((\forall p \in Pred_\Lang)(\forall a_1 \in |\mathcal{A}|)\dots(\forall a_{\#p} \in |\mathcal{A}|) \newline
[<a_1, \dots, a_{\#p}> \; \in p^\mathcal{A} \longleftrightarrow \; <h(a_1), \dots, h(a_{\#p})> \; \in p^\mathcal{A}]\);
\end{itemize}

Забелжка: в сила е \(\{id_{|\mathcal{A}|}\} \subseteq Aut(\mathcal{A})\).

\end{document}