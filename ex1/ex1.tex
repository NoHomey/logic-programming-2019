\documentclass{article}[12pt]
\usepackage{amsmath,amsthm}
\usepackage{amssymb}
\usepackage{lipsum}
\usepackage{stmaryrd}
\usepackage[T1,T2A]{fontenc}
\usepackage[utf8]{inputenc}
\usepackage[bulgarian]{babel}
\usepackage[normalem]{ulem}
\usepackage{xcolor}

\newcommand{\Lang}{\mathcal{L}}

\setlength{\parindent}{0mm}

\title{Синтаксис и семантика на език на предикатното смятане от първи ред}
\author{Иво Стратев}

\begin{document}
\maketitle

\tableofcontents

\vspace{1cm}

Предстои да бъде въведен един формален език.
Този език ще има напълно недвусмислен синтаксис и ясна семантика.
Една от целите ни с такъв език ще бъде да описваме (говорим) и работим в различни светове.
Мета език е езика, който използваме за да опишем формалния (обектния) език.

\section{Забележки}

Смятам, че използването на различи цветове ще помогне в разбирането и осмислянето на нещата,
защото ще послужи за ясно разграничаване между обектен език и мета език.
За това ще използва три цвята вместо един.

\begin{itemize}
\item В \color{cyan}син \color{black}цвят ще бъдат оцветявани буквите (символитите) от азбуката/ите.
Тоест например \(\color{cyan}{(}\) ще бъде буквата лява отваряща кръгла скоба.
\item В \color{red}червен \color{black}цвят ще бъдат оцветявани символи,
чието значние ще бъде или символ или редици от бувки (символи).
Целта е да е по-ясно, че използвайки тези символи за строене на думи ще имаме предвид,
че те биват заместени синтактично в строените думи или че значението на даден обект е символ.
Така например ако \(\color{red}\varphi \color{black}= \color{cyan}f(x)\),
то \(\color{cyan}s(\color{red}\varphi\color{cyan}, c) \color{black} = \color{cyan} s(f(x), c)\).
Ако пък кажем \(\color{red} \alpha \color{black} \in \{ \color{cyan} a \color{black}, \color{cyan} b \color{black}, \dots \} \),
то ще разбираме, че \(\color{red} \alpha\) е някой от символите (буквите) \(\color{cyan} a \color{black} , \color{cyan} b \color{black}, \dots\).
\item В черен цвят ще бъдат всички останали символи, които интерпретираме по стандартния за нас начин.
Тоест в черен цвят ще бъде всичко от мета езика, който за нас ще бъде българския език, смесен с езика на математиката,
който сме развили до момента (възприели до сега). 
\end{itemize}

Също така смятам, че използването на различни скоби би улеснило
разчитането на "дълги" формули, когато не е въведен приоритет на съждителните връзки.

\vspace{0.5cm}

И така сега ще кажем какво значи един език \(\Lang\) да бъде език на предикатното смятане от първи ред.

\section{Синтаксис}

\subsection{Логическа част}

Състои се от следните азбуки (множества от символи):

\begin{itemize}
\item азбука на съждителните връзки
\(\{\color{cyan}\lnot \color{black}, \color{cyan} \lor \color{black} , \color{cyan} \& \color{black} , \color{cyan}\implies \color{black} , \color{cyan}\iff \color{black} \}\);
\item азбука на кванторите
\(\{\color{cyan}\exists \color{black}, \color{cyan} \forall \color{black} \}\);
\item азбука на помощните символи
\(\{\color{cyan} [ \color{black}, \color{cyan} < \color{black} , \color{cyan} ( \color{black} , \color{cyan} , \color{black} , \color{cyan} ) \color{black} , \color{cyan} > \color{black} , \color{cyan} ] \color{black} \}\);
\item азбука на (обектовите/индивидните) променливи \(Var_\Lang\) \\
(изброимо множество от символи). 
Обикновенно за нас елементи на това множество ще бъдат буквите \(\color{cyan} x \; y \; z \; w\) с или без индекси.
Например напълно възможно е \(Var_\Lang = \{\color{cyan} x \color{black}, \color{cyan} y \color{black} , \color{cyan} z \color{black} , \color{cyan} x_\pi \color{black} , \color{cyan} y_{\frac{325}{66}} \color{black} , \color{cyan} z_{\epsilon} \color{black} \}\);
\item езика \(\Lang\) е възможно да съдържа, но не е задължително да съдържа символ \(\color{cyan}\doteq\),
който ще наричаме символ за формално равенство.
\end{itemize}

\subsection{Нелогическа част}

Състои се от следните азбуки (множества от символи, всяко от които е изброимо):

\begin{itemize}
\item множество на константните символи \(Const_\Lang\)
Обикновенно за нас елементи на това множество ще бъдат буквите \(\color{cyan} a \; b \; c \; d \) с или без индекси.
Например напълно възможно е \(Const_\Lang = \{\color{cyan} a \color{black}, \color{cyan} b \color{black} , \color{cyan} c \color{black} , \color{cyan} d_\pi \color{black} , \color{cyan} a_{2} \color{black} , \color{cyan} b_{\epsilon} \color{black} \}\);
\item множество на функционалните символи \(Func_\Lang\)
Обикновенно за нас елементи на това множество ще бъдат буквите \(\color{cyan} f \; g \; h \) с или без индекси.
Например напълно възможно е \(Func_\Lang = \{\color{cyan} f \color{black}, \color{cyan} g \color{black} , \color{cyan} h \color{black} , \color{cyan} f_\pi \color{black} , \color{cyan} g_{2} \color{black} , \color{cyan} h_{\epsilon} \color{black} \}\);
\item множество на предикатите символи \(Pred_\Lang\)
Обикновенно за нас елементи на това множество ще бъдат буквите \(\color{cyan} p \; q \; r \; t \) с или без индекси.
Например напълно възможно е \(Pred_\Lang = \{\color{cyan} p \color{black}, \color{cyan} q \color{black} , \color{cyan} r \color{black} , \color{cyan} t_\pi \color{black} , \color{cyan} p_{2} \color{black} , \color{cyan} q_{\epsilon} \color{black} \}\).
\end{itemize}

\subsection{Арност}

Арност ще наричаме функция \(\#_{(\_)} \; : \; Func_\Lang \cup Pred_\Lang \to \mathbb{N} \setminus \{0\} \).
Тази функция за нас неформално ще ни казва броя на аргументите на всеки функционален и предикатен символ.
Което като програмисти ще значи, че забраняваме претоварването (overloading) на символите нещо, с което сме свикнали от езика C++ .
Това е с цел наше улеснение, не обратното :) .

\subsection{Сигнатура}

Сигнатура на езика \(\Lang\) ще наричаме множеството
\begin{align*}
Const_\Lang \cup Func_\Lang \cup Pred_\Lang \; ( \; \cup \; \{ \color{cyan} \doteq  \color{black} \} \; )
\end{align*}

Може би най-близкото за нас нещо до сигнатура е интерфейс (абстрактен клас) (Type class) от програмирането.

\subsection{Пример до тук}

Нека \(\Lang\) включва формално равенство и още нека:

\begin{itemize}
\item \(Const_\Lang = \{\color{cyan}a \color{black}, \color{cyan} b \color{black} \}\);
\item \(Func_\Lang = \{\color{cyan}f \color{black}, \color{cyan} g \color{black} \}\);
\item \(Pred_\Lang = \emptyset\).
\end{itemize}

Света, в който бихме искали да работим (да опишем) е някой конкретен пръстен.
Тогава \(\#\color{cyan}f \color{black} = \#\color{cyan}g \color{black} = 2\) и сигнатурата в случая е множеството
\(\{\color{cyan}a \color{black}, \color{cyan} b \color{black} , \color{cyan} f \color{black} , \color{cyan} g \color{black} , \color{cyan} \doteq \color{black} \}\).

Предстои ни да се сблъскаме с понятието терм,
което не е нищо друго от име за обект от света,
в който желаем да работим/опишем.
Какво е терм ще дефинираме индуктивно.

\subsection{Терм}

База:

\begin{itemize}
\item \((\forall \color{red} \chi \color{black} \in Var_\Lang)[\color{red} \chi \color{black} \in Term_\Lang]\) (всяки символ за променлива е терм);
\item \((\forall \color{red} c \color{black} \in Const_\Lang)[\color{red} c \color{black} \in Term_\Lang]\) (всяки символ за константа е терм);
\end{itemize}

Имаме единствено правило чрез което да конструираме нови термове от вече получени такива.

\vspace{0.5cm}

Ако \(\color{red} \delta \color{black} \in Func_\Lang\) и \(\color{red}\tau_1\) е терм, \(\dots\), \(\color{red}\tau_{\color{black}\#\color{red}\delta}\) е терм, то
\(\color{red} \delta \color{cyan} (\color{red}\tau_1 \color{cyan} , \color{black} \dots \color{cyan} , \color{red} \tau_{\color{black}\#\color{red}\delta} \color{cyan} )\) също е терм.
(символичното прилагане на функционален символ към термове е терм).

\vspace{0.5cm}

Термове получени чрез горното правило могат да бъдат получени само след краен брой пъти неговото прилагане.

\vspace{0.5cm}

Множеството на термовете от езика \(\Lang\) ще означаваме с \(Term_\Lang\).

\subsubsection{Примери}

Надграждаме примера, който разглеждахме като искаме \(Var_\Lang = \{\color{cyan} x \color{black}, \color{cyan} y \color{black} , \color{cyan} z \color{black} \}\).
Тогава следните редици от символи са термове:

\begin{itemize}
\item \(\color{cyan}a\);
\item \(\color{cyan}b\);
\item \(\color{cyan}x\);
\item \(\color{cyan}y\);
\item \(\color{cyan}f(a, b)\);
\item \(\color{cyan}f(a, g(b, b))\);
\item \(\color{cyan}f(f(g(b, a), b), g(a, a))\).
\end{itemize}

\subsection{Атомарна формула}

Ако \(\color{red}\rho \color{black} \in Pred_\Lang\) и \(\color{red}\tau_1\), \(\dots\), \(\color{red}\tau_{\color{black}\#\color{red}\rho}\) са термове,
то \(\color{red}\rho \color{cyan} < \color{red} \tau_1 \color{cyan} , \color{black} \dots \color{cyan} , \color{red} \tau_{\color{black}\#\color{red}\rho} \color{cyan} >\)
е атомарна формула.

\vspace{0.5cm}

Ако \(\Lang\) е с формално равенство и \(\color{red} \alpha\) е терм и \(\color{red} \beta\) е терм,
то \(\color{cyan} < \color{red} \alpha \color{cyan} \doteq \color{red} \beta \color{cyan} >\) също е атомарна формула.

\vspace{0.5cm}

Сега ще дефинираме напълно формално, какво ще означава за нас формула.

\subsection{Формула}

База: Всяка атомарна формула е формула.

Можем да конструираме нови формули от други формули
чрез краен брой прилагания на следите три правила:

\begin{itemize}
\item Ако \(\color{red} \varphi\) е формула,
то \(\color{cyan} [ \neg \color{red} \varphi \color{cyan} ]\) е формула;
\item Ако \(\color{red} \varphi\) е формула,
\(\color{red} \psi\) е формула и \(\color{red} \sigma \color{black} \in \{\color{cyan} \lor \color{black} , \color{cyan} \& \color{black} , \color{cyan}\implies \color{black} , \color{cyan}\iff \color{black} \}\),
то \(\color{cyan} [ \color{red} \varphi \sigma \psi \color{cyan} ]\) е формула;
\item Ако \(\color{red} \varphi\) е формула, \(\color{red} \chi \color{black} \in Var_\Lang\) и \(\color{red} \Lambda \color{black} \in \{\color{cyan} \exists \color{black} , \color{cyan} \forall \color{black} \}\),
то \(\color{cyan} [ \color{red} \Lambda \chi \varphi \color{cyan} ]\) е формула.
\end{itemize}

Множеството от формули от езика \(\Lang\) ще бележим с \(\mathcal{F}_\Lang\).

\subsubsection{Примери}

Надграждаме още примера за пръстен. Следните думи са формули:

\begin{itemize}
\item \(\color{cyan} < x \doteq x >\);
\item \(\color{cyan} < a \doteq y >\);
\item \(\color{cyan} [\lnot < a \doteq y > ]\);
\item \(\color{cyan} [ < f(a, b) \doteq x > \& < g(x, y) \doteq y > ]\);
\item \(\color{cyan} [\forall x < x \doteq x > ]\);
\item \(\color{cyan} [\exists x < x \doteq x > ]\);
\item \(\color{cyan} [ [\forall x < x \doteq x > ] \implies [ < f(g(a, y), x) \doteq x > \& < g(x, y) \doteq y > ] ]\);
\item \(\color{cyan} [\forall x [\exists y <f(x, y) \doteq a>]]\).
\end{itemize}

Така вече можем да кажем, че думите от език на предикатното смятане от първи са формулите от този език.

\section{Семантика}

Няма как да дефинираме семантика без да кажем какво значи всеки символ от езика
(как да го интерпретираме).
И така първо ще дефиираме понятието структура.

\subsection{Структура}
Структура за нас ще бъде наредена двойка \((A, I)\), за която

\begin{itemize}
\item \(A\) е непразно множество, което ще наримаче универсум, домей или носител на структурата;
\item \(I\) ще е изображение, което ще наричаме интерпретация;
\item Ако \(\color{red} \epsilon \color{black} \in Const_\Lang\),
то \(I(\color{red} \epsilon \color{black}) \in A\) (интерпретацията на всеки символ за константа е фиксиран елемент на носителя);
\item Ако \(\color{red} \pi \color{black} \in Func_\Lang\),
то \(I(\color{red} \pi \color{black}) \; : A^{\#\color{red} \pi} \color{black} \to A\) (интерпретацията на всеки функционален символ е функция между носителя на степен арността на функционалния символ и носителя);
\item Ако \(\color{red} \rho \color{black} \in Pred_\Lang\),
то \(I(\color{red} \rho \color{black}) \; \subseteq A^{\#\color{red} \rho}\) (интерпретацията на всеки предикатен символ е релация с домейн носителя на степен арността на предикатния символ);
\end{itemize}

Неформално можем да кажем, че интерпретацията на всеки:

\begin{itemize}
\item символ за константа е константа;
\item функционален символ е функция;
\item предикатен символ е релация.
\end{itemize}

Обикновенно за структурите ще ползваме голяма ръкописна буква,
а за носителя/универсума на структурата голяма печатна буква.

\vspace{0.3cm}

Тоест в случая \(\mathcal{A} = (A, I)\).

\vspace{0.3cm}

И ако имаме структура \(\mathcal{A}\),
то нейния носител ще означаваме с \(|\mathcal{A}|\).

\vspace{0.3cm}

Също така ако \(\color{red} \xi \color{black} \in Const_\Lang \cup Func_\Lang \cup Pred_\Lang\),
то вместо \(I(\color{red} \xi \color{black})\) ще ползваме означението \(\color{red}\xi \color{black} ^{\mathcal{A}}\).

\subsubsection{Пример}

Разширяваме още примера, който разглеждаме.

Една структура за еика, който разглеждаме е следната:

\begin{itemize}
\item \(\mathcal{A} = (\mathbb{Z}, I)\);
\item \(I(\color{cyan} a \color{black}) = \color{cyan} a \color{black}^{\mathcal{A}} = 0_\mathbb{Z}\);
\item \(I(\color{cyan} b \color{black}) = \color{cyan} b \color{black}^{\mathcal{A}} = 1_\mathbb{Z}\);
\item \(I(\color{cyan} f \color{black}) = \color{cyan} f \color{black}^{\mathcal{A}} = +_\mathbb{Z}\);
\item \(I(\color{cyan} g \color{black}) = \color{cyan} g \color{black}^{\mathcal{A}} = ._\mathbb{Z}\).
\end{itemize}

За да дефинираме семантиката на \(Lang\) ще ни трябва да кажем какво значи да оценим една променлива.
Идея, с която до болка сме привикнали когато пишем програми.
Надявам се всеки е запознат с начина за оценяване в езика Scheme (Racket) - модела на средите.
Ще се опитаме да формализираме този модел.

\subsection{Оценка}

Оценка в универсума на една структура \(\mathcal{A}\) на променливите от езика \(\Lang\)
ще наричаме функция \(\nu \; : \; Var_\Lang \to |\mathcal{A}| \).

\vspace{0.3cm}

Нека разширим оценката, до функция над термове.

\subsection{Стойност на терм}

Нека рекурсивно дефинираме разширената функция

\(\$ \; (\_) \; \$^{\mathcal{A}}_\nu \; : \; Term_\Lang \to |\mathcal{A}|\)
по следния начин:

\begin{itemize}
\item \((\forall \color{red} \chi \color{black} \in Var_\Lang)[\; \$ \; \color{red} \chi \color{black} \; \$^{\mathcal{A}}_\nu \overset{def}{=} \nu(\color{red} \chi \color{black}) \; ]\);
\item \((\forall \color{red} \epsilon \color{black} \in Const_\Lang)[\; \$ \; \color{red} \epsilon \color{black} \; \$^{\mathcal{A}}_\nu \overset{def}{=} \color{red} \epsilon \color{black}^{\mathcal{A}} \; ]\);
\item Ако \(\color{red} \pi \color{black} \in Func_\Lang, \color{red}  \tau_1 \color{black} \in Term_\Lang, \dots, \color{red} \tau_{\color{black} \# \color{red} \pi} \color{black} \in Term_\Lang\), то \\
\(\$ \; \color{red} \pi  \color{cyan} ( \color{red} \tau_1 \color{cyan} , \color{black} \dots \color{cyan} , \color{red} \tau_{\color{black} \# \color{red} \pi} \color{cyan} ) \color{black} \; \$^{\mathcal{A}}_\nu \overset{def}{=} \color{red} \pi \color{black}^{\mathcal{A}}( \$ \; \color{red} \tau_1 \color{black} \; \$^{\mathcal{A}}_\nu, \dots , \$ \; \color{red} \tau_{\color{black} \# \color{red} \pi} \color{black} \; \$^{\mathcal{A}}_\nu  )\).
\end{itemize}

\subsubsection{Пример}
Нека пресметнем стойостта на следния терм
\begin{align*}
\color{cyan} g(f(b, x), y)
\end{align*}

в структурата \(\mathcal{A}\) от предния пример,
при оценка \(\nu \; : \; \{\color{cyan} x \color{black}, \color{cyan} y \color{black} , \color{cyan} z \color{black} \} \to \mathbb{Z}\),
за която \(\nu(\color{cyan} x \color{black}) = 3, \; \nu(\color{cyan} y \color{black}) = -5, \; \nu(\color{cyan} z \color{black}) = 7\).

\begin{align*}
\$ \; \color{cyan}g(f(b, x), y) \color{black} \; \$^{\mathcal{A}}_\nu = \\
\color{cyan} g \color{black} ^{\mathcal{A}}(\$ \; \color{cyan} f(b, x) \color{black} \; \$^{\mathcal{A}}_\nu, \$ \; \color{cyan} y \color{black} \; \$^{\mathcal{A}}_\nu ) = \\
._\mathbb{Z}(\color{cyan} f \color{black} ^{\mathcal{A}}(\$ \; \color{cyan} b \color{black} \; \$^{\mathcal{A}}_\nu, \$ \; \color{cyan} x \color{black} \; \$^{\mathcal{A}}_\nu ), \nu(\color{cyan} y \color{black})) = \\
._\mathbb{Z}(+_\mathbb{Z}(\color{cyan} b \color{black} ^{\mathcal{A}} , \nu(\color{cyan} x \color{black}), -5) = \\
._\mathbb{Z}(+_\mathbb{Z}(1,  3) , -5) =  \\
._\mathbb{Z}(4, -5) = \\
-20
\end{align*}

\subsection{Модифицирана оценка}

За да разберем, защо ще се нуждаем от следващата дефиниция
нека разгледаме следния израз на езика Scheme
\begin{align*}
\color{blue}
((lambda \; (x) \; (+ \; x \; ((lambda \; (x) \; (* \; x \; 3)) \; 2))) \; 5)
\end{align*}

Горният израз по стойност е еквивалентен със следния математически
\begin{align*}
(x + (3x)(2))(5)
\end{align*}

И сега може би е по-ясно, че при пресмятането на израза \(3x\),
е в сила \(x = 2\), а във външното пресмятане е в сила \(x = 5\). 

\vspace{0.5cm}

Ако сега разгледаме формулата
\begin{align*}
\color{cyan}
[[\exists x <x \doteq b>] \; \& \; [\exists x <x \doteq a>]]
\end{align*}

Без да сме въвели какво значи семантика на формула нека кажем неформално,
че искаме семантиката на една формула да е нейната вярност в света, в който я разглеждаме.

\vspace{0.5cm}

И така ако искаме разглежданата формула да е вярна в пръстена, който образуват целите числа,
тоест в света, който разглеждаме в примерите.
То бихме искали да е истина следното изречение на български:
Същесвува цяло число \(x\), равно на цялото число 1 и същесвува цяло число \(x\), равно на цялото число 0.
Ясно е че това е вярно и в първия случай \(x\) е 1 във втория 0.

\vspace{0.5cm}

Ясно е, че се нуждаем от начин да "свързваме" стойност с променлива при разглеждането на формула или подформула. 
Тоест трябва ни начин, по който да моделираме промяна на оценката на дадена формула.

\vspace{0.5cm}

Въвеждаме понятието модифицирана оценка.
Ако имаме една оценка \(\nu\) и променлива \(\color{red} \chi\) и искаме нейната оценка да е \(a\),
който е фиксиран елемент на универсума на структурата, която разглеждаме.
То \(\nu^{\color{red} \chi}_{\color{black} a}\) е модифицираната оценка на \(\nu\),
която оценява \(\color{red} \chi\) с \(a\). Тоест

\begin{align*}
\nu^{\color{red} \chi}_{\color{black} a} (\color{red} \gamma \color{black}) = \begin{cases}
 a , \;\; \color{red} \gamma \color{black} = \color{red} \chi \color{black} \\
\nu(\color{red} \gamma \color{black}) , \;\; \color{red} \gamma \color{black} \neq \color{red} \chi \color{black}
\end{cases}
\end{align*}

\subsubsection{Пример}

Ако модефицираме стойността на променливата \(\color{cyan} y\) да бъде \(5\) в предния пример,
ще получим

\begin{align*}
\displaystyle{\$ \; \color{cyan}g(f(b, x), y) \color{black} \; \$^{\mathcal{A}}_{\nu^{\color{cyan} y}_{\color{black} 5}}} = 20
\end{align*}

\subsection{Една последна дефиниця преди въвеждането на Семантика}
Нека означаваме константите "истина" с И и "лъжа" с Л.


Нека \(h_{\neg}, h_{\lor}, h_{\&}, h_{\implies}, h_{\iff}\) са стандартните булеви функции: негация, дизюнкция, конюнкция, импликация и биимпликация.

И нека \(Eval \; : \; \{\color{cyan}\lnot \color{black}, \color{cyan} \lor \color{black} , \color{cyan} \& \color{black} , \color{cyan}\implies \color{black} , \color{cyan}\iff \color{black} \} \to \{h_{\neg}, h_{\lor}, h_{\&}, h_{\implies}, h_{\iff}\} \),
която на символ съпоставя съответната булева функция. Тоест

\begin{align*}
Eval(\color{cyan} \neg \color{black}) = h_{\neg} \\
Eval(\color{cyan} \lor \color{black}) = h_{\lor} \\
Eval(\color{cyan} \& \color{black}) = h_{\&} \\
Eval(\color{cyan} \implies \color{black}) = h_{\implies} \\
Eval(\color{cyan} \iff \color{black}) = h_{\iff}
\end{align*}

\subsection{Семантика на формула в структура \(\mathcal{A}\) за езика (отговаряща на сигнатурата на езика) при оценка \(\nu\)}

Дефинираме функция \( \| \; (\_) \; \|^{\mathcal{A}}_{\nu} \; : \; \mathcal{F}_\Lang \to \{\text{И}, \text{Л}\} \) по следните правила:

\begin{itemize}
\item Ако \(\rho \color{black} \in Pred_\Lang, \color{red} \tau_1 \color{black} \in Term_\Lang, \dots, \color{red} \tau_{\color{black}\#\color{red}\rho} \color{black} \in Term_\Lang\), то \\
\(\| \; \color{red}\rho \color{cyan} < \color{red} \tau_1 \color{cyan} , \color{black} \dots \color{cyan} , \color{red} \tau_{\color{black}\#\color{red}\rho} \color{cyan} > \color{black} \; \|^{\mathcal{A}}_{\nu} = \text{И} \iff (\$ \; \color{red} \tau_1 \color{black}  \; \$^{\mathcal{A}}_{\nu}, \dots  , \$ \; \color{red} \tau_{\color{black}\# \color{red} \rho} \color{black}  \; \$^{\mathcal{A}}_{\nu} ) \in \color{red} \rho\color{black}^{\mathcal{A}}\).
\item Ако \(\Lang\) е с формално равенство и \(\color{red} \tau \color{black} \in Term_\Lang\) и \(\color{red} \gamma \color{black} \in Term_\Lang\), то \\
\(\| \; \color{cyan} < \color{red} \tau \color{cyan} \doteq \color{red} \gamma \color{cyan} > \color{black} \; \|^{\mathcal{A}}_{\nu} = \text{И} \iff \$ \; \color{red} \tau \color{black} \; \$^{\mathcal{A}}_{\nu} = \$ \; \color{red} \gamma \color{black} \; \$^{\mathcal{A}}_{\nu}\).
\item Ако \(\color{red} \varphi \color{black} \in \mathcal{F}_\Lang\), то \(\| \; \color{cyan} [ \neg \color{red} \varphi \color{cyan} ] \color{black} \; \|^{\mathcal{A}}_{\nu} = h_\neg(\| \; \color{red} \varphi \color{black} \; \|^{\mathcal{A}}_{\nu})\).
\item Ако \(\color{red} \varphi \color{black} \in \mathcal{F}_\Lang, \color{red} \sigma \color{black} \in \{\color{cyan} \lor \color{black} , \color{cyan} \& \color{black} , \color{cyan}\implies \color{black} , \color{cyan}\iff \color{black} \}, \color{red} \psi \color{black} \in \mathcal{F}_\Lang\), то \\
\(\| \; \color{cyan} [ \color{red} \varphi \sigma \psi \color{cyan} ] \color{black} \; \|^{\mathcal{A}}_{\nu} = Eval(\color{red} \sigma \color{black})(\| \; \color{red} \varphi \color{black} \; \|^{\mathcal{A}}_{\nu}, \| \; \color{red} \psi \color{black} \; \|^{\mathcal{A}}_{\nu})\).
\item Ако \(\color{red} \varphi \color{black} \in \mathcal{F}_\Lang, \color{red} \chi \color{black} \in Var_\Lang\), то \\
\(\| \; \color{cyan} [ \exists \color{red} \chi \varphi \color{cyan} ] \color{black} \; \|^{\mathcal{A}}_{\nu} = \text{И} \iff \text{съществува } a \in |\mathcal{A}| \text{ и } \| \; \color{red} \varphi \color{black} \; \|^{\mathcal{A}}_{\nu^{\color{red} \chi \color{black}}_a} = \text{И}\).
\item Ако \(\color{red} \varphi \color{black} \in \mathcal{F}_\Lang, \color{red} \chi \color{black} \in Var_\Lang\), то \\
\(\| \; \color{cyan} [ \forall \color{red} \chi \varphi \color{cyan} ] \color{black} \; \|^{\mathcal{A}}_{\nu} = \text{И} \iff \text{за всеки елемент } a \in |\mathcal{A}| \text{ е в сила } \| \; \color{red} \varphi \color{black} \; \|^{\mathcal{A}}_{\nu^{\color{red} \chi \color{black}}_a} = \text{И}\).
\end{itemize}

\subsection{Модел}

Казваме, че структура \(\mathcal{A}\) и оценка \(\nu\) са модел за формула \(\color{red} \varphi\)
тогава и само тогава когато \(\| \; \color{red} \varphi  \color{black}\; \|^{\mathcal{A}}_{\nu} = \text{И}\).
Ако \(\mathcal{A}\) и оценка \(\nu\) са модел за \(\color{red} \varphi\), то  ще го означаваме с \(\mathcal{A} \underset{\nu}{\models} \color{red} \varphi\).

\subsection{Примери}

Нека проверим дали структурата и оценката от примера, който разглеждаме са модели за:
\begin{itemize}
\item \(\color{cyan} < x \doteq y > \);
\item \(\color{cyan} [ \exists z < f(x, z) \doteq a > ] \);
\item \(\color{cyan} [\exists x \; [ \forall y \; [ \exists z < f(y, z) \doteq x >] ]]\)
\end{itemize}

\subsubsection{Формула 1}
\(\| \; \color{cyan} < x \doteq y > \color{black}\; \|^{\mathcal{A}}_{\nu} = \text{И} \iff \$ \; \color{cyan} x \color{black} \; \$^{\mathcal{A}}_{\nu} = \$ \; \color{cyan} y \color{black} \; \$^{\mathcal{A}}_{\nu} \iff 3 = -5 \iff \text{Л} \),
Следователно \(\mathcal{A}\) и \(\nu\) не са модел.

\subsubsection{Формула 2}
\begin{align*}
\| \; \color{cyan} [ \exists z < f(x, z) \doteq a > ] \color{black}\; \|^{\mathcal{A}}_{\nu} = \text{И} \iff \\
\text{ същесвува } \alpha \in \mathbb{Z} \text{ и } \| \; \color{cyan} < f(x, z) \doteq a > \color{black}\; \|^{\mathcal{A}}_{\nu^{\color{cyan} z \color{black}}_\alpha}  = \text{И} \iff \\
\text{ същесвува } \alpha \in \mathbb{Z} \text{ и } \$ \; \color{cyan} f(x, z) \color{black}\; \$^{\mathcal{A}}_{\nu^{\color{cyan} z \color{black}}_\alpha} = \$ \; \color{cyan} a \color{black}\; \$^{\mathcal{A}}_{\nu^{\color{cyan} z \color{black}}_\alpha} \iff \\
\text{ същесвува } \alpha \in \mathbb{Z} \text{ и } 3 + \alpha = 0 \iff \\
3 + (-3) = 0 \iff \text{И}
\end{align*}

Тоест избирайки \(\alpha = -3\) получаваме истина, което означава че същесвува елемент \(\alpha\) с желаното свойство.
Следователно \(\mathcal{A} \underset{\nu}{\models} \color{cyan} [ \exists z < f(x, z) \doteq a > ]\)

\subsubsection{Формула 3}
Проверете сами, че \(\mathcal{A} \underset{\nu}{\models} \color{cyan} [\exists x \; [ \forall y \; [ \exists z < f(y, z) \doteq x >] ]]\).  =)

\end{document}